%%%%%%%%%%%%%%%%%%%%%%%%%%%%%%%%%%%%%%%%%%%%%%%%%%%%%%%
% A template for Wiley article submissions.
% Developed by Overleaf. 
%
% Please note that whilst this template provides a 
% preview of the typeset manuscript for submission, it 
% will not necessarily be the final publication layout.
%
% Usage notes:
% The "blind" option will make anonymous all author, affiliation, correspondence and funding information.
% Use "num-refs" option for numerical citation and references style.
% Use "alpha-refs" option for author-year citation and references style.

\documentclass[alpha-refs]{wiley-article}
% \documentclass[blind,num-refs]{wiley-article}

% Add additional packages here if required
\usepackage{siunitx}

% Update article type if known
\papertype{Original Article}
% Include section in journal if known, otherwise delete
%\paperfield{Journal Section}

\title{Camila Title Suggestions?}

% List abbreviations here, if any. Please note that it is preferred that abbreviations be defined at the first instance they appear in the text, rather than creating an abbreviations list.
\abbrevs{ABC, a black cat; DEF, doesn't ever fret; GHI, goes home immediately.}

% Include full author names and degrees, when required by the journal.
% Use the \authfn to add symbols for additional footnotes and present addresses, if any. Usually start with 1 for notes about author contributions; then continuing with 2 etc if any author has a different present address.
\author[1\authfn{1}]{Author One PhD}
\author[2\authfn{1}]{Author A.~Two MD}
\author[2\authfn{2}]{Author Three PhD}
\author[2]{Author B.~Four}

\contrib[\authfn{1}]{Equally contributing authors.}

% Include full affiliation details for all authors
\affil[1]{Department, Institution, City, State or Province, Postal Code, Country}
\affil[2]{Department, Institution, City, State or Province, Postal Code, Country}

\corraddress{Author One PhD, Department, Institution, City, State or Province, Postal Code, Country}
\corremail{correspondingauthor@email.com}

\presentadd[\authfn{2}]{Department, Institution, City, State or Province, Postal Code, Country}

\fundinginfo{Funder One, Funder One Department, Grant/Award Number: 123456, 123457 and 123458; Funder Two, Funder Two Department, Grant/Award Number: 123459}

% Include the name of the author that should appear in the running header
\runningauthor{Author One et al.}

\begin{document}

\maketitle

\begin{abstract}
This is a generic template designed for use by multiple journals, which includes several options for customization. Please consult the author guidelines for the journal to which you are submitting in order to confirm that your manuscript will comply with the journal's requirements. Please replace this text with your abstract.

% Please include a maximum of seven keywords
\keywords{keyword 1, \emph{keyword 2}, keyword 3, keyword 4, keyword 5, keyword 6, keyword 7}
\end{abstract}

\section{Introduction}
Coming Soon...

\section{Materials and Methods}

To improve trap catch and monitoring of adult onion maggot fly, iterations of sticky traps of various shape, size, and color were evaluated in 2005 and 2006 in Upstate New York.  Additionally, after an improved shape, size, and color trap was developed, this trap was tested in conjunction with a Delia Lure attractant to evaluate potential improvement in trap catch.  

All shape, size, and color trials were conducted in commercial fields on muck soil near Elba, NY in western upstate New York (\textbf{GPS coordinates, other information about unique muck characteristics?}).  Trials evaluating trap catch in conjunction with Delia Lure were evaluated both in Elba, NY and on muck soil near Potter, NY in western upstate New York.  No modifications were done to commercial weed, disease, or pest management and spray programs (which closely followed recommended management \citep{reiners2011integrated}), the primary target of which is \textit{Thrips tabaci}.  This management is unlikely to affect \textit{D. antiqua} populations due to spray timing and minimal impacts of thrips sprays on onion maggot fly \citep{finch1986behavior}. 

\subsection{Field Trials}

\paragraph{Shape Trial} To evaluate effect of trap shape on trap catch and monitoring of adult onion maggot fly, square, cube, short cylindrical, tall cylindrical, and spherical semi-gloss white shapes each with a surface area of 180 $cm^2$  were coated with sticky material (\textbf{Tanglefoot?}) and placed along the long edge of an onion (\textit{Allium cepa} L.) field near Elba, NY.  Traps were spaced 15.24 meters apart and arranged in a randomized control block design with five replications.  Traps were placed in the field in late May then collected in late June (\textbf{Not sure about this, more info needed?}).   

\paragraph{Size Trial}
To evaluate the influence of trap size on the trap catch and monitoring of adult onion maggot fly, white spherical sticky traps with diameters of 5.00, 6.25, 7.50, 8.75, and 10.00 cm diameter were placed along the long edge of onion fields with spacing 15.24 meters apart in a randomized controlled block design with five replications.  

\paragraph{Color Trial}
To evaluate the effect of color on trap catch and monitoring of adult onion maggot fly, white, yellow, green, and red spherical sticky traps 8.75 cm in diameter were placed along the long edge of onion fields in a randomized controlled block design with five replications.  

\paragraph{Delia Lure Trial}
To evaluate the effect of adding an attractant an trap catch and monitoring of adult onion maggot flies, white spherical sticky traps 8.75 cm in diameter were placed at the edge of onion fields in a paired design where half of the traps received a Delia Lure (Baited) and the other half did not (Unbaited).  Trap Catch was monitored three times throughout the season and four replications of each treatment were conducted in each location.  


\subsection{Analysis}

Raw data from shape, size, and color trials were analyzed using linear models and analysis of variance.  Models were determined after consideration of all factor combinations and interactions and selected based on consideration of residual diagnostics (conformance to assumptions of normality and homoscedasticity), goodness of fit tests, $R^2$ values, information criteria, and leverage considerations. Based on these considerations, an outlier was detected and removed from the shape trials (P = 0.032, Bonferroni Outlier Test) and data from the size and color trials were square root transformed to conform with assumptions of normality and homoscedasticity.    

Raw data from the Delia Lure Trial were analyzed with a paired t-test due to the nature of the experimental design.  Conformation to assumptions of normality were confirmed through examination of quantile-quantile plots.

\subsection{Data Management}

All data for the trials were entered into flat tabular (.csv) files.  All analysis on the raw data was conducted in R version 3.5.2 using RStudio as an IDE (with Vim keybindings) \citep{rcore2018,rstudio}.  The $tidyverse$, $car$, and $emmeans$ packages were used to facilitate analysis \citep{tidy, car, emmeans}.  All code, including manuscript documentation, is available on GitHub (https://github.com/acetworld/onion-maggot-attraction).

\section{Results}


\paragraph{Shape Trial} Models relating trap shape and \textit{D. antiqua} sex to trap catch significantly explained approximately 70\% of observed variation (P \textless 0.001, $R^2_{adj}$ = 0.70, Table \ref{table:1}).  Trap shape significantly influenced (P = 0.005) catch of adult \textit{D. antiqua} (Figure \ref{fig:figure1}).  Sticky traps caught approximately 27.81 $\pm$ 2.8 more female flies than male flies.  Short cylindrical traps outperformed square panel and tall cylinder traps in terms of female trap catch.  No significant differences were observed between male trap catch.  

\begin{table}[]
\caption{Linear model and analysis of variance results and diagnostics for shape, size, and color trials.  Treatment refers to the factor under evaluation. For the shape trial, treatment denotes shape; for the size trial, treatment denotes size; for the color trial, treatment denotes color.  Missing values indicate that factor was not included in the best fit model.  All code and additional documentation is available on GitHub.  }
\begin{tabular}{lrrrrrrrrr}
              & \multicolumn{9}{c}{Trial}                                                                                                                                                                                                \\ \cline{2-10} 
              & \multicolumn{3}{c}{Shape}                                              & \multicolumn{3}{c}{Size}                                               & \multicolumn{3}{c}{Color}                                              \\
              & \multicolumn{1}{c}{F} & \multicolumn{1}{c}{df} & \multicolumn{1}{c}{P} & \multicolumn{1}{c}{F} & \multicolumn{1}{c}{df} & \multicolumn{1}{c}{P} & \multicolumn{1}{c}{F} & \multicolumn{1}{c}{df} & \multicolumn{1}{c}{P} \\ \cline{2-10}
Sex           & 42                    & 1                      & \textless 0.0001      & 197.5                 & 1                      & \textless 0.0001      & 37.2                  & 1                      & \textless 0.0001      \\
Treatment     & 4.4                   & 4                      & 0.005                 & 11.6                  & 4                      & \textless 0.0001      & 31.3                  & 3                      & \textless 0.0001      \\
Year          &                       &                        &                       & 142.4                 & 1                      & \textless 0.0001      &                       &                        &                       \\
Sex:Treatment &                       &                        &                       &                       &                        &                       & 6.3                   & 3                      & 0.002                 \\
Sex:Year      &                       &                        &                       & 137.9                 & 1                      & \textless 0.0001      &                       &                        &                       \\ \cline{2-10} 
              &                       &                        &                       &                       &                        &                       &                       &                        &                       \\
$R^2$            &                       &                        & 0.75                  &                       &                        & 0.75                  &                       &                        & 0.83                  \\
$R^2_{adj}$         &                       &                        & 0.70                  &                       &                        & 0.73                  &                       &                        & 0.79                  \\
P             &                       &                        & \textless 0.0001      &                       &                        & \textless 0.0001      &                       &                        & \textless 0.0001     
\end{tabular}
\label{table:1}
\end{table}



\paragraph{Size Trial} Models using size, sex, and year to explain observed trap catch significantly explained approximately 73\% of the observed variation (P \textless 0.001, $R^2_{adj}$ = 0.73, Table \ref{table:1}).  Trap size significantly influenced trap catch (P \textless 0.001).  Increasing diameter of trap size showed a trend towards increasing trap catch (Figure \ref{fig:figure2}) with the largest trap catching significantly more females and males.


\paragraph{Color Trial} Models using color and sex to explain observed trap catch significantly explained approximately 79\% of the observed variation (P \textless 0.001, $R^2_{adj}$ = 0.79, Table \ref{table:1}). Color significantly influenced trap catch (P \textless 0.001) with white traps catching significantly more females than any other color (t \textless -3.1, df = 32, P \textless 0.012) and significantly more males than any other color except yellow (t \textless -3.1, df = 32, P \textless 0.012, Yellow: t = -1.6, df = 32, P = 0.3).  White traps caught significantly more females than males (t = 6.1, df = 32, P \textless 0.001).

\paragraph{Delia Lure Trial} Traps baited with Delia Lure caught approximately 51.3 (95\% CI: 7.3-95.4) more adult flies than traps without the Delia Lure bait (t = 3.0, df = 5, P = 0.03).




\begin{figure}[bt]
\centering
\includegraphics[width = 8cm]{figures/publication/figure-1.pdf}
\caption{Adult onion maggot fly catch on sticky traps of different shape.  Shaded points denote raw values while solid points and error bars denote mean and bootstrapped 95\% confidence intervals respectively.  Letters not shared between groups denote significant differences in female trap catch (there was no observed difference in male trap catch) at $\alpha$ = 0.05 with Tukey's HSD test. }
\label{fig:figure1}
\end{figure}

\begin{figure}[bt]
\centering
\includegraphics[width = 8cm]{figures/publication/figure-2.pdf}
\caption{Adult onion maggot fly catch on spherical sticky traps of different size.  Points and error bars denote mean and 95\% confidence intervals respectively.  Letters not shared between groups indicates significant differences in adult catch at $\alpha$ = 0.05 with Tukey's HSD test.}
\label{fig:figure2}
\end{figure}

\begin{figure}[bt]
\centering
\includegraphics[width = 8cm]{figures/publication/figure-3.pdf}
\caption{Adult onion maggot fly catch on spherical sticky traps of different color.  Shaded points denote raw values while solid points and error bars denote means and bootstrapped 95\% confidence intervals respectively.  \textbf{*} indicates significantly different trap catch on white sticky traps (P = 0.012).}
\label{fig:figure3}
\end{figure}


\begin{figure}[bt]
\centering
\includegraphics[width = 8cm]{figures/publication/figure-4.pdf}
\caption{Adult onion maggot fly catch on sticky traps baited with Delia Lure attractant and unbaited.  Shaded points denote raw values while solid points and error bars denote means and bootstrapped 95\% confidence intervals respectively.  \textbf{*} indicates significantly different trap catch on baited and unbaited traps (P = 0.030). }
\label{fig:figure4}
\end{figure}


\section{Discussion}

\section*{Acknowledgements}
Acknowledgements should include contributions from anyone who does not meet the criteria for authorship (for example, to recognize contributions from people who provided technical help, collation of data, writing assistance, acquisition of funding, or a department chairperson who provided general support), as well as any funding or other support information.

\section*{Conflict of Interest}
You may be asked to provide a conflict of interest statement during the submission process. Please check the journal's author guidelines for details on what to include in this section. Please ensure you liaise with all co-authors to confirm agreement with the final statement.

\section*{Author Contribution}

\section*{Data Availability Statement}

\printendnotes

% Submissions are not required to reflect the precise reference formatting of the journal (use of italics, bold etc.), however it is important that all key elements of each reference are included.


\graphicalabstract{example-image-1x1}{Please check the journal's author guildines for whether a graphical abstract, key points, new findings, or other items are required for display in the Table of Contents.}

\end{document}
